\documentclass[a5paper]{article}
\usepackage[english]{babel}
\usepackage{graphicx}
\usepackage{multicol}
\usepackage{amsmath}
\usepackage{hyperref}
\usepackage{amsthm}
\usepackage{geometry}
\geometry{a5paper}
\usepackage{fancyhdr}
\usepackage{xcolor}
\usepackage{amssymb}
\usepackage{multicol}
\theoremstyle{definition}
\newtheorem{exmp}{Example}[section]
\newtheorem{theorem}{Theorem}

\begin{document}
\author{
    \textbf{Ibrahim Abou Elenein} 50-17004
  \and
    \textbf{Jack Amir} 50-2421
    \and
    \textbf{Moamen Nazmy} 50-19334
    \and
    \textbf{Somaya} 50-8504
}
\title{\textbf{Reference List}}
\maketitle
% \tableofcontents
% \section{Introduction}
\section{Article 1}
\begin{enumerate}
    \item
Airaksinen O, Brox JI, Cedraschi C, et al. Chapter 4: European guidelines for the
management of chronic nonspecific low back pain. Eur Spine J. 2006;15(Suppl.
2):S192–S300.

\item
Hoy D, Bain C, Williams G, et al. Rheumatism. 2012;64(6):2028–2037.



\item
Kuijer W, Brouwer S, Preuper HR, et al. Work status and chronic low back pain:
exploring the International Classification of Functioning, Disability and Health.
Disabil Rehabil. 2006;28:379–388.


\item
Richardson CA, Jull GA, Hodges PW, Hides JA. Therapeutic Exercise for Spinal
Segmental Stabilization in LBP: Scientific Basis and Clinical Approach. Edinburgh:
Churchill Livingstone; 1999


\item
Hayden JA, van Tulder MW, Malmivaara A, et al. Exercise therapy for treatment of
non-specific low back pain. Cochrane Database Syst Rev. 2005;20.


\item
Yamato TP, Maher CG, Saragiotto BT, et al. Pilates for Low Back Pain. Cochrane
Database of Systematic Reviews. 2015; 2015 Issue 7. Art. No.: CD010265. 10.1002/
14651858. CD010265. pub2.


\item
Wells C, Kolt GS, Bialocerkowski A. Defining Pilates exercise: a systematic review.
Complement Ther Med. 2012;20:253–262.


\item
Critchley DJ, Pierson Z, Batters G. Effect of pilates mat exercises and conventional
exercise programmes on transversus abdominis and obliquus internus abdominis activity: pilot randomised trial. Man Ther. 2011;16(April (2)):183–189.


\item
Whittaker JL, Thompson JA, Teyhen DS, Hodges P. Rehabilitative ultrasound imaging
of pelvic floor muscle function. J Orthop Sports Phys Ther. 2007;37(August
(8)):487–498.


\item
Critchley DJ, Pierson Z, Batters G. Effect of pilates mat exercises and conventional
exercise programmes on transversus abdominis and obliquus internus abdominis
activity: pilot randomised trial. Man Ther. 2011;16(April (2)):183–189.


\item
Oliveira NTB, SMSF Freitas, Moura KF, Junior MAL. Cabral CMN Biomechanical
analysis of the trunk and pelvis during pilates method exercises: systematic review.
Fisioter Pesqui. 2015;22(4):443–455 10.590/1809-2950/14068822042015.


\item
Chan L, Heinemann AW, Roberts J. Elevating the quality of disability and rehabilitation research: mandatory use of the reporting guidelines. Ann Phys Rehabil
Med. 2014;57:558–560.


\item
Jensen MP, Karoly P. Self-report scales and procedures for assessing pain inadults. In:
Turk DC, Melzack R, eds. Handbook of Pain Assessment. New York: Guilford Press;
2001:15–34.


\item
Roland M, Morris R. A study of the natural history of back pain, part I: the development of a reliable and sensitive measure of disability in low back pain. Spine.
1983;8:141–144.


\item
Hodges PW. Is there a role for transversus abdominis in lumbo-pelvic stability? Man
Ther. 1999;4(May (2)):74–86.


\item
Hides JA, Miokovic T, Belavy DL, et al. Ultrasound imaging assessment of abdominal
muscle function during drawing-in of the abdominal wall: an intrarater reliability
study. J Orthop Sports Phys Ther. 2007;37:480–486.


\item
Chen YH, Chai HM, Shau YW, et al. Increased sliding of transverse abdominis during
contraction after myofascial release in patients with chronic low back pain. Man Ther
Jun. 2016;23:69–75.


\item
Kim KH, Cho SH, Goo BO, et al. Differences in transversus abdominis muscle function
between chronic low back pain patients and healthy subjects at maximum expiration:
measurement with real-time ultrasonography. J Phys Ther Sci. 2013;25(July
(7)):861–863.


\item
French DJ, France CR, Vigneau F, et al. Fear of movement/(re)injury in chronic pain:
a psychometric assessment of the original English version of the Tampa Scale for
Kinesiophobia (TSK). Pain. 2007;127:42–51.


\item
Vlaeyen JWS, Kole-Snijders AMJ, Rotteveel AM, et al. The role of fear of movement/
(re)injury in pain disability. J Occup Rehabil. 1995;5:235–252.


\item
Swinkels-Meewisse E, Swinkels R, Verbeek A, et al. Psychometric properties of the
Tampa Scale for Kinesiophobia and the fear avoidance beliefs questionnaire in acute
low back pain. Man Ther. 2003;8:29–36.

\item
Natour J, Cazotti LA, Ribeiro LH, et al. Pilates improves pain, function and quality of
life in patients with chronic low back pain: a randomized controlled trial. Clin Rehabil.
2015;29:59–68.


\item
Cruz-Díaz D, Martínez-Amat A, Osuna-Pérez MC, et al. Short- and long-term effects of
a six-week clinical Pilates program in addition to physical therapy on postmenopausal
women with chronic low back pain: a randomized controlled trial. Disabil Rehabil.
2016;38(June (13)):1300–1308.


\item
Blum CL. Chiropractic and Pilates therapy for the treatment of adult scoliosis. J
Manipulative Physiol Ther. 2002;25:E3.


\item
Kamioka H, Tsutani K, Katsumata Y. Effectiveness of Pilates exercise: a quality evaluation and summary of systematic reviews based on randomized controlled trials.
Complement Ther Med. 2016;25(April):1–19. http://dx.doi.org/10.1016/j.ctim.2015.
12.01.


\item
Wajswelner H, Metcalf B, Bennell K. Clinical pilates versus general exercise for
chronic low back pain: randomized trial. Med Sci Sports Exerc. 2012;44:1197–1205.


\item
Cruz-Díaz D, Martínez-Amat A, De la Torre-Cruz MJ, et al. Effects of a six-week Pilates
intervention on balance and fear of falling in women aged over 65 with chronic lowback pain: a randomized controlled trial. Maturitas. 2015;82(December (4)):371–376.


\item
Harrington L, Davies R. The influence of Pilates training on the ability to contract the
Transversus abdominis muscle in asymptomatic individuals. J Body Work Mov Ther.
2005;9:52–57.



\item
Grooms DR, Grindstaff TL, Croy T, et al. Clinimetric analysis of pressure biofeedback
and transversus abdominis function in individuals with stabilization classification low
back pain. J Orthop Sports Phys Ther. 2013;43(March (3)):184–193.


\item
Critchley DJ, Coutts FJ. Abdominal muscle function in chronic low back pain patients:
measurement with real-time ultrasound scanning. Physiotherapy. 2002;88(322):e32.


\item
Hodges PW. Core stability exercise in chronic low back pain. Orthop Clin North Am.
2003;34(April (2)):245–254.

\item

da Luz Jr, MA, Costa LO, Fuhro FF, Manzoni AC, Oliveira NT, Cabral CM.
Effectiveness of mat Pilates or equipment-based Pilates exercises in patients with
chronic nonspecific low back pain: a randomized controlled trial. Phys Ther.
2014;94(5):623–631. http://dx.doi.org/10.2522/ptj.20130277 [Epub 2014 Jan 16].

\item

Lee CW, Hyun J, Kim SG. Influence of pilates mat and apparatus exercises on pain and
balance of businesswomen with chronic low back pain. J Phys Ther Sci. 2014;26(April
(4)):475–477. http://dx.doi.org/10.1589/jpts.26.475.

\item
Hodges PW. Changes in sensorimotor control in low back pain. J Electromyogr Kinesiol.
2011;21(April (2)):220–228.
\end{enumerate}
\section{Article 2}
1. Bergström C, Jensen I, Hagberg J, Busch H, Bergström G. Effectiveness
of different interventions using a psychosocial subgroup assignment
in chronic neck and back pain patients: A 10‑year follow‑up. Disabil
Rehabil 2012;34:110‑8.

2. Hoy DG, Protani M, De R, Buchbinder R. The epidemiology of neck
pain. Best Pract Res Clin Rheumatol 2010;24:783‑92.

3. Balagué F, Mannion AF, Pellisé F, Cedraschi C. Non‑specific low back
pain. Lancet 2012;379:482‑91.

4. Sadock BJ, Sadock VA. Kaplan and Sadock’s Synopsis of Psychiatry:
Behavioral Sciences/Clinical Psychiatry. New York: Lippincott
Williams \& Wilkins; 2011.

5. Al‑Obaidi SM, Al‑Sayegh NA, Ben Nakhi H, Al‑Mandeel M.
Evaluation of the McKenzie intervention for chronic low back pain by
using selected physical and bio‑behavioral outcome measures. PM R
2011;3:637‑46.

6. DehkordiAH, Heydarnejad MS. Effect of booklet and combined method
on parents’ awareness of children with beta‑thalassemia major disorder.
J Pak Med Assoc 2008;58:485‑7.

7. van der Wees PJ, Jamtvedt G, Rebbeck T, de Bie RA, Dekker J,
Hendriks EJ. Multifaceted strategies may increase implementation
of physiotherapy clinical guidelines: A systematic review. Aust J
Physiother 2008;54:233‑41.

8. Maas ET, Juch JN, Groeneweg JG, Ostelo RW, Koes BW, Verhagen AP,
et al. Cost‑effectiveness of minimal interventional procedures for chronic
mechanical low back pain: Design of four randomised controlled trials
with an economic evaluation. BMC Musculoskelet Disord 2012;13:260.

9. Hernandez AM, Peterson AL. Work‑related musculoskeletal disorders
and pain. Handbook of Occupational Health and Wellness. Springer;
2012. p. 63‑85.

10. Hassanpour Dehkordi A, Khaledi Far A. Effect of exercise training on
the quality of life and echocardiography parameter of systolic function
in patients with chronic heart failure: A randomized trial. Asian J Sports
Med 2015;6:e22643.

11. Hasanpour‑Dehkordi A, Khaledi‑Far A, Khaledi‑Far B, Salehi‑Tali S.
The effect of family training and support on the quality of life and cost
of hospital readmissions in congestive heart failure patients in Iran. Appl
Nurs Res 2016;31:165‑9.

12. Hassanpour Dehkordi A. Influence of yoga and aerobics exercise on
fatigue, pain and psychosocial status in patients with multiple sclerosis:
A Randomized Trial. J Sports Med Phys Fitness 2015; [Epub ahead of
print].

13. Hassanpour‑Dehkordi A, Jivad N. Comparison of regular aerobic and
yoga on the quality of life in patients with multiple sclerosis. Med J
Islam Repub Iran 2014;28:141.

14. Heydarnejad S, Dehkordi AH. The effect of an exercise program on the
health‑quality of life in older adults. A randomized controlled trial. Dan
Med Bull 2010;57:A4113.

15. van Middelkoop M, Rubinstein SM, Verhagen AP, Ostelo RW,
Koes BW, van Tulder MW. Exercise therapy for chronic nonspecific
low‑back pain. Best Pract Res Clin Rheumatol 2010;24:193‑204.

16. Critchley DJ, Pierson Z, Battersby G. Effect of pilates mat exercises
and conventional exercise programmes on transversus abdominis and
obliquus internus abdominis activity: Pilot randomised trial. Man Ther
2011;16:183‑9.

17. Kloubec JA. Pilates for improvement of muscle endurance, flexibility,
balance, and posture. J Strength Cond Res 2010;24:661‑7.

18. Hosseinifar M, Akbari A, Shahrakinasab A. The effects of McKenzie
and lumbar stabilization exercises on the improvement of function and
pain in patients with chronic low back pain: A randomized controlled
trial. J Shahrekord Univ Med Sci 2009;11:1‑9.

19. GarciaAN, Costa Lda C, da Silva TM, Gondo FL, Cyrillo FN, Costa RA,
et al. Effectiveness of back school versus McKenzie exercises in patients
with chronic nonspecific low back pain: A randomized controlled trial.
Phys Ther 2013;93:729‑47.

20. Hassanpour‑Dehkordi A, Safavi P, Parvin N. Effect of methadone
maintenance treatment of opioid dependent fathers on mental health and
perceived family functioning of their children. Heroin Addict Relat Clin
2016;18(3):9-14.

21. Shahbazi K, Solati K, Hasanpour‑Dehkordi A. Comparison of
hypnotherapy and standard medical treatment alone on quality of life
in patients with irritable bowel syndrome: A Randomized Control Trial.
J Clin Diagn Res 2016;10:OC01‑4.

22. Ngamkham S, Vincent C, Finnegan L, Holden JE, Wang ZJ, Wilkie DJ.
The McGill Pain Questionnaire as a multidimensional measure in people
with cancer: An integrative review. Pain Manag Nurs 2012;13:27‑51.

23. Sterling M. General health questionnaire‑28 (GHQ‑28). J Physiother
2011;57:259.

24. Petersen T, Kryger P, Ekdahl C, Olsen S, Jacobsen S. The effect of
McKenzie therapy as compared with that of intensive strengthening
training for the treatment of patients with subacute or chronic low back
pain: Arandomized controlled trial. Spine (Phila Pa 1976) 2002;27:1702‑9.

25. Gladwell V, Head S, Haggar M, Beneke R. Does a program of
pilates improve chronic non‑specific low back pain? J Sport Rehabil
2006;15:338‑50.

26. Udermann BE, Mayer JM, Donelson RG, Graves JE, Murray SR.
Combining lumbar extension training with McKenzie therapy: Effects
on pain, disability, and psychosocial functioning in chronic low back
pain patients. Gundersen Lutheran Med J 2004;3:7‑12.

27. Machado LA, Maher CG, Herbert RD, Clare H, McAuley JH. The
effectiveness of the McKenzie method in addition to first‑line care
for acute low back pain: A randomized controlled trial. BMC Med
2010;8:10.

28. Kilpikoski S. The McKenzie Method in Assessing, Classifying and
Treating Non‑Specific Low Back Pain in Adults with Special Reference to
the Centralization Phenomenon. Jyväskylä: University of Jyväskylä; 2010.

29. Borges J, Baptista AF, Santana N, Souza I, Kruschewsky RA,
Galvão‑Castro B, et al. Pilates exercises improve low back pain and
quality of life in patients with HTLV‑1 virus: A randomized crossover
clinical trial. J Bodyw Mov Ther 2014;18:68‑74.

30. Caldwell K, Harrison M, Adams M, Triplett NT. Effect of pilates and
taiji quan training on self‑efficacy, sleep quality, mood, and physical
performance of college students. J Bodyw Mov Ther 2009;13:155‑63.

31. Altan L, Korkmaz N, Bingol U, Gunay B. Effect of pilates training
on people with fibromyalgia syndrome: A pilot study. Arch Phys Med
Rehabil 2009;90:1983‑8.
\section{Article 3}
Balagué F, Mannion AF, Pellisé F, Cedraschi C. Non-specific low back
pain. Lancet 2012;379:482-491.

Biering-Sørensen F. A prospective study of low back pain in a general
population. I. Occurrence, recurrence and aetiology. Scand J Rehabil
Med 1983;15:71-79.

Bindra S, Sinha AK, Benjamin AI. Epidemiology of low back pain in Indian population: a review. Int J Appl Basic Med Res 2015;5:166-179.

Burton AK, Balagué F, Cardon G, Eriksen HR, Henrotin Y, Lahad A, Leclerc A, Müller G, van der Beek AJ; COST B13 Working Group on
Guidelines for Prevention in Low Back Pain. Chapter 2. European
guidelines for prevention in low back pain: November 2004. Eur
Spine J 2006;15 Suppl 2:S136-168.

Carlsson AM. Assessment of chronic pain. I. Aspects of the reliability and
validity of the visual analogue scale. Pain 1983;16:87-101.
Comfort P, Pearson SJ, Mather D. An electromyographical comparison of
trunk muscle activity during isometric trunk and dynamic strengthening exercises. J Strength Cond Res 2011;25:149-154.

de Paula Lima PO, de Oliveira RR, Costa LO, Laurentino GE. Measurement properties of the pressure biofeedback unit in the evaluation of
transversus abdominis muscle activity: a systematic review. Physiotherapy 2011;97:100-106.

Deyo RA, Walsh NE, Martin DC, Schoenfeld LS, Ramamurthy S. A controlled trial of transcutaneous electrical nerve stimulation (TENS) and
exercise for chronic low back pain. N Engl J Med 1990;322:1627-1634.

Ferreira PH, Ferreira ML, Maher CG, Herbert RD, Refshauge K. Specific
stabilisation exercise for spinal and pelvic pain: a systematic review.

Aust J Physiother 2006;52:79-88.
Fritz JM, Irrgang JJ. A comparison of a modified Oswestry Low Back Pain
Disability Questionnaire and the Quebec Back Pain Disability Scale.
Phys Ther 2001;81:776-788.

Frymoyer JW, Cats-Baril WL. An overview of the incidences and costs of
low back pain. Orthop Clin North Am 1991;22:263-271.

Gladwell V, Head S, Haggar M, Beneke R. Does a program of Pilates improve chronic non-specific low back pain? J Sport Rehabil 2006;15:338-
350.

Graven-Nielsen T, Svensson P, Arendt-Nielsen L. Effects of experimental
muscle pain on muscle activity and co-ordination during static and
dynamic motor function. Electroencephalogr Clin Neurophysiol
1997;105:156-164.

Han TS, Schouten JS, Lean ME, Seidell JC. The prevalence of low back
pain and associations with body fatness, fat distribution and height.
Int J Obes Relat Metab Disord 1997;21:600-607.

Hayden JA, van Tulder MW, Malmivaara A, Koes BW. Exercise therapy
for treatment of non-specific low back pain. Cochrane Database Syst
Rev 2005;(3):CD000335.

Koes BW, van Tulder MW, Thomas S. Diagnosis and treatment of low
back pain. BMJ 2006;332:1430-1434.
Koumantakis GA, Watson PJ, Oldham JA. Trunk muscle stabilization
training plus general exercise versus general exercise only: randomized controlled trial of patients with recurrent low back pain. Phys
Ther 2005;85:209-225.

Krismer M, van Tulder M; Low Back Pain Group of the Bone and Joint
Health Strategies for Europe Project. Strategies for prevention and
management of musculoskeletal conditions. Low back pain (non-specific). Best Pract Res Clin Rheumatol 2007;21:77-91.

Macedo LG, Maher CG, Latimer J, McAuley JH. Motor control exercise
for persistent, nonspecific low back pain: a systematic review. Phys
Ther 2009;89:9-25.

Manniche C, Lundberg E, Christensen I, Bentzen L, Hesselsøe G. Intensive dynamic back exercises for chronic low back pain: a clinical trial.
Pain 1991;47:53-63.

Moon HJ, Choi KH, Kim DH, Kim HJ, Cho YK, Lee KH, Kim JH, Choi YJ.
Effect of lumbar stabilization and dynamic lumbar strengthening exercises in patients with chronic low back pain. Ann Rehabil Med
2013;37:110-117.

Nadler SF, Steiner DJ, Erasala GN, Hengehold DA, Hinkle RT, Beth
Goodale M, Abeln SB, Weingand KW. Continuous low-level heat
wrap therapy provides more efficacy than Ibuprofen and acetaminophen for acute low back pain. Spine (Phila Pa 1976) 2002;27:1012-1017.

Pereira LM, Obara K, Dias JM, Menacho MO, Guariglia DA, Schiavoni D,
Pereira HM, Cardoso JR. Comparing the Pilates method with no exercise or lumbar stabilization for pain and functionality in patients with
chronic low back pain: systematic review and meta-analysis. Clin Rehabil 2012;26:10-20.

Poitras S, Brosseau L. Evidence-informed management of chronic low
back pain with transcutaneous electrical nerve stimulation, interferential current, electrical muscle stimulation, ultrasound, and thermotherapy. Spine J 2008;8:226-233.

Rydeard R, Leger A, Smith D. Pilates-based therapeutic exercise: effect on
subjects with nonspecific chronic low back pain and functional disability: a randomized controlled trial. J Orthop Sports Phys Ther
2006;36:472-484.

Saal JA, Saal JS. Nonoperative treatment of herniated lumbar interverte-
bral disc with radiculopathy. An outcome study. Spine (Phila Pa 1976)
1989;14:431-437.

Sculco AD, Paup DC, Fernhall B, Sculco MJ. Effects of aerobic exercise on
low back pain patients in treatment. Spine J 2001;1:95-101.
Sherman KJ, Cherkin DC, Erro J, Miglioretti DL, Deyo RA. Comparing
yoga, exercise, and a self-care book for chronic low back pain: a randomized, controlled trial. Ann Intern Med 2005;143:849-856.

Shiri R, Karppinen J, Leino-Arjas P, Solovieva S, Viikari-Juntura E. The
association between obesity and low back pain: a meta-analysis. Am J
Epidemiol 2010;171:135-154.

Storheim K, Bø K, Pederstad O, Jahnsen R. Intra-tester reproducibility of
pressure biofeedback in measurement of transversus abdominis function. Physiother Res Int 2002;7:239-249.

Tousignant M, Poulin L, Marchand S, Viau A, Place C. The Modified-Modified Schober Test for range of motion assessment of lumbar
flexion in patients with low back pain: a study of criterion validity, intra- and inter-rater reliability and minimum metrically detectable
change. Disabil Rehabil 2005;27:553-559.

Zambito A, Bianchini D, Gatti D, Viapiana O, Rossini M, Adami S. Interferential and horizontal therapies in chronic low back pain: a randomized, double blind, clinical study. Clin Exp Rheumatol 2006;24:534-539.
Zanoli G, Strömqvist B, Jönsson B. Visual analog scales for interpretation
of back and leg pain intensity in patients operated for degenerative
lumbar spine disorders. Spine (Phila Pa 1976) 2001;26:2375-2380.
\section{Article 4}
1. Airaksinen O, Brox C, Cedraschi C, et al. European Guidelines for
the Management of Chronic Non-Specific Low Back Pain. On
behalf of the COST B13 Working Group on Guidelines for
Chronic Low Back Pain [Internet]. 2005 [cited 2009 Sep 9].
Guidelines.pdf.

2. Anderson B. Randomised clinical trial comparing active versus
passive approaches to the treatment of recurrent and chronic low
back pain [dissertation]. Miami (FL): University of Miami; 2005.
p. 209.

3. Crow R, Gage H, Hampson S, Hart J, Kimber A, Thomas H. The
role of expectancies in the placebo effect and their use in the delivery of health care: a systematic review. Health Technol Assess.
1999;3(3):1–96.

4. Davidson M, Keating JL. A comparison of five low back disability
questionnaires: reliability and responsiveness. Phys Ther. 2002;
82(1):8–24.

5. Donzelli S, Di Domenica E, Cova AM, Galletti R, Giunta N. Two
different techniques in the rehabilitation treatment of low back
pain: a randomized controlled trial. Eura Medicophys. 2006;42(3):
205–10.

6. Ferreira ML, Ferreira PH, Latimer J, et al. Comparison of general exercise, motor control exercise and spinal manipulative
therapy for chronic low back pain: a randomized trial. Pain.
2007;131(1–2):31–7.

7. Fersum KV, Dankaerts W, O’Sullivan PB, et al. Integration of
subclassification strategies in randomised controlled clinical trials
evaluating manual therapy treatment and exercise therapy for nonspecific chronic low back pain: a systematic review. Br J Sports
Med. 2010;44(14):1054–62.

8. Gagnon L. Efficacy of Pilates exercises as a therapeutic intervention in treating patients with low back pain [dissertation]. Knoxville (TN): University of Tennessee; 2005. p. 119.

9. George SZ, Wittmer VT, Fillingim RB, Robinson ME. Comparison of graded exercise and graded exposure clinical outcomes for
patients with chronic low back pain. J Orthop Sports Phys Ther.
2010;40(11):694–704.

10. Geraets JJ, de Groot IJ, Goossens ME, et al. Comparison of two
recruitment strategies for patients with chronic shoulder complaints. Br J Gen Pract. 2006;56(523):127–33.

11. Gladwell V, Head S, Haggar M, Beneke R. Does a program of
Pilates improve chronic non-specific low back pain. J Sport
Rehabil. 2006;15:338–50.

12. Guzman J, Esmail R, Karjalainen K, Malmivaara A, Irvin E,
Bombardier C. Multidisciplinary bio-psycho-social rehabilitation
for chronic low-back pain. Cochrane Database Syst Rev. 2006;
(2):CD000963.
13. Hall H, McIntosh G. Low back pain (chronic). Clin Evid (Online)
[Internet]. 2008 [cited 2908004]; Accessed August 18, 2011.

14. Hayden JA, van Tulder MW, Malmivaara A, Koes BW. Exercise
therapy for treatment of non-specific low back pain. Cochrane
Database Syst Rev. 2005;(3):CD000335.

15. Henchoz Y, Kai-Lik So A. Exercise and nonspecific low back
pain: a literature review. Joint Bone Spine. 2008;75(5):533–9


16. Henschke N, Maher CG, Refshauge KM, et al. Prognosis in
patients with recent onset low back pain in Australian primary
care: inception cohort study. BMJ. 2008;337:a171.

17. Hicks GE, Manal TJ. Psychometric properties of commonly used
low back disability questionnaires: are they useful for older adults
with low back pain? Pain Med. 2009;10(1):85–94.

18. Kamper SJ, Ostelo RW, Knol DL, Maher CG, de Vet HC,
Hancock MJ. Global Perceived Effect scales provided reliable
assessments of health transition in people with musculoskeletal
disorders, but ratings are strongly influenced by current status.
J Clin Epidemiol. 2010;63(7):760–6.e1.

19. Koes BW, van Tulder MW, Thomas S. Diagnosis and treatment of
low back pain. BMJ. 2006;332(7555):1430–4.

20. Kopec JA, Esdaile JM, Abrahamowicz M, et al. The Quebec Back
Pain Disability Scale. Measurement properties. Spine (Phila Pa
1976). 1995;20(3):341–52.

21. Krogsboll LT, Hrobjartsson A, Gotzsche PC. Spontaneous improvement in randomised clinical trials: meta-analysis of threearmed trials comparing no treatment, placebo and active intervention. BMC Med Res Methodol. 2009;9:1.

22. La Touche R, Escalante K, Linares MT. Treating non-specific
chronic low back pain through the Pilates Method. J Bodyw Mov
Ther. 2008;12(4):364–70.

23. Lambeek LC, Bosmans JE, Van Royen BJ, Van Tulder MW,
Van Mechelen W, Anema JR. Effect of integrated care for sick
listed patients with chronic low back pain: economic evaluation
alongside a randomised controlled trial. BMJ. 2010;341:c6414.

24. Leijon O, Mulder M. Prevalence of low back pain and concurrent
psychological distress over a 16-year period. Occup Environ Med.
2009;66(2):137–9.

25. Lim EC, Poh RL, Low AY, Wong WP. Effects of Pilates-based
exercises on pain and disability in individuals with persistent
nonspecific low back pain: a systematic review with meta-analysis.
J Orthop Sports Phys Ther. 2011;41(2):70–80.

26. Mannion AF, Balague F, Pellise F, Cedraschi C. Pain measurement
in patients with low back pain. Nat Clin Pract Rheumatol. 2007;
3(11):610–8.

27. Morton V, Torgerson DJ. Regression to the mean: treatment effect
without the intervention. J Eval Clin Pract. 2005;11(1):59–65.

28. Nicholas MK. The Pain Self-efficacy Questionnaire: taking pain
into account. Eur J Pain. 2007;11(2):153–63.

29. O’Brien N, Hanlon M, Meldrum D. Randomised, controlled trial
comparing physiotherapy and Pilates in the treatment of ordinary
low back pain. Phys Ther Rev. 2006;11:224–5.

30. Ostelo RW, Deyo RA, Stratford P, et al. Interpreting change scores
for pain and functional status in low back pain: towards international consensus regarding minimal important change. Spine
(Phila Pa 1976). 2008;33(1):90–4.

31. Pereira LM, Obara K, Dias JM, et al. Comparing the Pilates method
with no exercise or lumbar stabilization for pain and functionality in
patients with chronic low back pain: systematic review and metaanalysis. Clin Rehabil [Internet]. Accessed August 18, 2011 [cited].

32. Quinn J. Influence of Pilates-based mat exercises on chronic lower
back pain [dissertation]. Boca Raton {FL}: Florida Atlantic University; 2005. p. 64.
\end{document}
